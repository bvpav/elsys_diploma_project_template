\section{Общи насоки за техническо писане}

Тези точки се отнасят за цялата дипломна работа, не само за тази глава.

\subsection{Фигури}
\emph{Винаги} се позовавайте на включените фигури, като Фигура \cref{fig:relu}, в основния текст. Включвайте пълни, обяснителни надписи и се уверете, че фигурите изглеждат добре на страницата.
Можете да включите множество фигури в един float, както във Фигура \cref{fig:synthetic}, използвайки \texttt{subcaption}, който е активиран в шаблона.


% Figures are important. Use them well.
\begin{figure}[htb]
    \centering
    \includegraphics[width=0.5\linewidth]{images/relu.pdf}    

    \caption{В надписите на фигурите обяснете какво вижда читателят: ,,Схема на изправящата линейна единица, където $a$ е изходната амплитуда,
    $d$ е конфигурируема мъртва зона, а $Z_j$ е входният сигнал'', както и защо читателят гледа това: 
    ,,Забележително е, че няма никаква активация \emph{изобщо} под 0, което обяснява първоначалните ни резултати.'' 
    \textbf{Използвайте векторни формати на изображения (.pdf) където е възможно}. Оразмерявайте фигурите подходящо и не ги правете прекалено големи или твърде малки за четене.
    }

    % use the notation fig:name to cross reference a figure
    \label{fig:relu} 
\end{figure}


\begin{figure}[htb] 
    \centering
    \begin{subfigure}[h]{0.45\textwidth}
        \includegraphics[width=\textwidth]{images/synthetic.png}
        \caption{Синтетично изображение, черно на бяло.}
        \label{fig:syn1}
    \end{subfigure}
    ~ %add desired spacing between images, e. g. ~, \quad, \qquad, \hfill etc. 
      %(or a blank line to force the subfigure onto a new line)
    \begin{subfigure}[h]{0.45\textwidth}
        \includegraphics[width=\textwidth]{images/synthetic_2.png}
        \caption{Синтетично изображение, бяло на черно.}
        \label{fig:syn2}
    \end{subfigure}
    ~ %add desired spacing between images, e. g. ~, \quad, \qquad, \hfill etc. 
    %(or a blank line to force the subfigure onto a new line)    
    \caption{Синтетични тестови изображения за алгоритми за откриване на ръбове. \subref{fig:syn1} показва различни нива на сиво, които изискват адаптивен алгоритъм. \subref{fig:syn2}
    показва по-предизвикателни тестове за откриване на ръбове, които имат пресичащи се линии. Сливането им в пълни сегменти обикновено изисква алгоритми като трансформацията на Хаф.
    Това е пример за използване на подфигури, с \texttt{subref}s в надписа.
    }\label{fig:synthetic}
\end{figure}

\clearpage

\subsection{Уравнения}

Уравненията трябва да бъдат набрани правилно и прецизно. Уверете се, че размерът на скобите е правилен и пунктуацията на уравненията е правилна 
(запетаята е важна и отива \textit{вътре} в блока на уравнението). Обяснете ясно всички използвани символи, ако не са дефинирани по-рано. 

Например, бихме могли да дефинираме:
\begin{equation}
    \hat{f}(\xi) = \frac{1}{2}\left[ \int_{-\infty}^{\infty} f(x) e^{2\pi i x \xi} \right],
\end{equation}    
където $\hat{f}(\xi)$ е Фурие трансформацията на времевия сигнал $f(x)$.

\subsection{Алгоритми}
Алгоритмите могат да бъдат набрани с помощта на \texttt{algorithm2e}, както в Алгоритъм \cref{alg:metropolis}.

% NOTE: line ends are denoted by \; in algorithm2e
\begin{algorithm}
    \DontPrintSemicolon
    \KwData{$f_X(x)$, a probability density function returing the density at $x$.\; $\sigma$ a standard deviation specifying the spread of the proposal distribution.\;
    $x_0$, an initial starting condition.}
    \KwResult{$s=[x_1, x_2, \dots, x_n]$, $n$ samples approximately drawn from a distribution with PDF $f_X(x)$.}
    \Begin{
        $s \longleftarrow []$\;
        $p \longleftarrow f_X(x)$\;
        $i \longleftarrow 0$\;
        \While{$i < n$}
        {
            $x^\prime \longleftarrow \mathcal{N}(x, \sigma^2)$\;
            $p^\prime \longleftarrow f_X(x^\prime)$\;
            $a \longleftarrow \frac{p^\prime}{p}$\;
            $r \longleftarrow U(0,1)$\;
            \If{$r<a$}
            {
                $x \longleftarrow x^\prime$\;
                $p \longleftarrow f_X(x)$\;
                $i \longleftarrow i+1$\;
                append $x$ to $s$\;
            }
        }
    }
    
\caption{Алгоритъмът Metropolis-Hastings MCMC за извличане на проби от произволни вероятностни разпределения, 
специализиран за нормални предложени разпределения $q(x^\prime|x) = \mathcal{N}(x, \sigma^2)$. Симетрията на нормалното разпределение означава, че правилото за приемане приема опростената форма.}\label{alg:metropolis}
\end{algorithm}

\subsection{Таблици}

Ако трябва да включите таблици, като Таблица \cref{tab:operators}, използвайте инструмент като\\ https://www.tablesgenerator.com/ за генериране на таблицата, тъй като е
изключително досадно в противен случай. 

\begin{table}
    \caption{Стандартната таблица с оператори в Python, заедно с техните функционални еквиваленти от пакета \texttt{operator}. Забележете, че надписите
    на таблиците отиват над таблицата, а не под нея. Не добавяйте допълнителни линии/черти към таблиците.}\label{tab:operators}
    %\tt 
    \rowcolors{2}{}{gray!3}
    \begin{tabular}{@{}lll@{}}
    %\toprule
    \textbf{Операция}    & \textbf{Синтаксис}                & \textbf{Функция}                            \\ %\midrule % optional rule for header
    Addition              & \texttt{a + b}                          & \texttt{add(a, b)}                                    \\
    Concatenation         & \texttt{seq1 + seq2}                    & \texttt{concat(seq1, seq2)}                           \\
    Containment Test      & \texttt{obj in seq}                     & \texttt{contains(seq, obj)}                           \\
    Division              & \texttt{a / b}                          & \texttt{div(a, b) }  \\
    Division              & \texttt{a / b}                          & \texttt{truediv(a, b) } \\
    Division              & \texttt{a // b}                         & \texttt{floordiv(a, b)}                               \\
    Bitwise And           & \texttt{a \& b}                         & \texttt{and\_(a, b)}                                  \\
    Bitwise Exclusive Or  & \texttt{a \textasciicircum b}           & \texttt{xor(a, b)}                                    \\
    Bitwise Inversion     & \texttt{$\sim$a}                        & \texttt{invert(a)}                                    \\
    Bitwise Or            & \texttt{a | b}                          & \texttt{or\_(a, b)}                                   \\
    Exponentiation        & \texttt{a ** b}                         & \texttt{pow(a, b)}                                    \\
    Identity              & \texttt{a is b}                         & \texttt{is\_(a, b)}                                   \\
    Identity              & \texttt{a is not b}                     & \texttt{is\_not(a, b)}                                \\
    Indexed Assignment    & \texttt{obj{[}k{]} = v}                 & \texttt{setitem(obj, k, v)}                           \\
    Indexed Deletion      & \texttt{del obj{[}k{]}}                 & \texttt{delitem(obj, k)}                              \\
    Indexing              & \texttt{obj{[}k{]}}                     & \texttt{getitem(obj, k)}                              \\
    Left Shift            & \texttt{a \textless{}\textless b}       & \texttt{lshift(a, b)}                                 \\
    Modulo                & \texttt{a \% b}                         & \texttt{mod(a, b)}                                    \\
    Multiplication        & \texttt{a * b}                          & \texttt{mul(a, b)}                                    \\
    Negation (Arithmetic) & \texttt{- a}                            & \texttt{neg(a)}                                       \\
    Negation (Logical)    & \texttt{not a}                          & \texttt{not\_(a)}                                     \\
    Positive              & \texttt{+ a}                            & \texttt{pos(a)}                                       \\
    Right Shift           & \texttt{a \textgreater{}\textgreater b} & \texttt{rshift(a, b)}                                 \\
    Sequence Repetition   & \texttt{seq * i}                        & \texttt{repeat(seq, i)}                               \\
    Slice Assignment      & \texttt{seq{[}i:j{]} = values}          & \texttt{setitem(seq, slice(i, j), values)}            \\
    Slice Deletion        & \texttt{del seq{[}i:j{]}}               & \texttt{delitem(seq, slice(i, j))}                    \\
    Slicing               & \texttt{seq{[}i:j{]}}                   & \texttt{getitem(seq, slice(i, j))}                    \\
    String Formatting     & \texttt{s \% obj}                       & \texttt{mod(s, obj)}                                  \\
    Subtraction           & \texttt{a - b}                          & \texttt{sub(a, b)}                                    \\
    Truth Test            & \texttt{obj}                            & \texttt{truth(obj)}                                   \\
    Ordering              & \texttt{a \textless b}                  & \texttt{lt(a, b)}                                     \\
    Ordering              & \texttt{a \textless{}= b}               & \texttt{le(a, b)}                                     \\
    % \bottomrule
    \end{tabular}
    \end{table}
\subsection{Код}

Избягвайте да поставяте големи блокове код в отчета (повече от страница в един блок, например). Използвайте оцветяване на синтаксиса, ако е възможно, както в Листинг \cref{lst:callahan}.

\begin{lstlisting}[language=python, float, caption={The algorithm for packing the $3\times 3$ outer-totalistic binary CA successor rule into a 
    $16\times 16\times 16\times 16$ 4 bit lookup table, running an equivalent, notionally 16-state $2\times 2$ CA.}, label=lst:callahan]
    def create_callahan_table(rule="b3s23"):
        """Generate the lookup table for the cells."""        
        s_table = np.zeros((16, 16, 16, 16), dtype=np.uint8)
        birth, survive = parse_rule(rule)

        # generate all 16 bit strings
        for iv in range(65536):
            bv = [(iv >> z) & 1 for z in range(16)]
            a, b, c, d, e, f, g, h, i, j, k, l, m, n, o, p = bv

            # compute next state of the inner 2x2
            nw = apply_rule(f, a, b, c, e, g, i, j, k)
            ne = apply_rule(g, b, c, d, f, h, j, k, l)
            sw = apply_rule(j, e, f, g, i, k, m, n, o)
            se = apply_rule(k, f, g, h, j, l, n, o, p)

            # compute the index of this 4x4
            nw_code = a | (b << 1) | (e << 2) | (f << 3)
            ne_code = c | (d << 1) | (g << 2) | (h << 3)
            sw_code = i | (j << 1) | (m << 2) | (n << 3)
            se_code = k | (l << 1) | (o << 2) | (p << 3)

            # compute the state for the 2x2
            next_code = nw | (ne << 1) | (sw << 2) | (se << 3)

            # get the 4x4 index, and write into the table
            s_table[nw_code, ne_code, sw_code, se_code] = next_code

        return s_table

\end{lstlisting}

%=====================================================================================================

\subsection{Доказателства}
Уверете се, че представяте доказателствата си добре. Използвайте подходящи визуализации, 
техники за докладване и статистически анализ, според случая. Целта не е
да изсипете всички данни, които имате, а да представите добре подкрепен с доказателства аргумент.

Ако използвате числови доказателства, посочете разумен брой значещи цифри; не казвайте ,,18.41141\% от потребителите са били успешни'', ако сте имали само 20 потребители. Ако усреднявате \textit{каквото и да е}, представете както мярка за централна тенденция (напр. средна стойност, медиана), \textit{така и} мярка за разпръскване (напр. стандартно отклонение, мин/макс, интерквартилен диапазон).

Можете да използвате \texttt{siunitx} за дефиниране на единици, подреждане на числата спретнато и задаване на прецизност за целия LaTeX документ. 

% setup siunitx to have two decimal places
\sisetup{
	round-mode = places,
	round-precision = 2
}

Например, тези числа ще се появят с два десетични знака: \num{3.141592}, \num{2.71828}, а това ще се появи с разумно разстояние \num{1000000}.

Ако използвате статистически процедури, уверете се, че разбирате процеса, който използвате,
и че проверявате дали необходимите предположения са валидни във вашия случай. 

Ако визуализирате, следвайте основните правила, както е илюстрирано във Фигура \cref{fig:boxplot}:
\begin{itemize}
\item Надпишете всичко правилно (оси, заглавие, единици).
\item Надпишете подробно.
\item Цитирайте в текста.
\item \textbf{Включете подходящо показване на несигурността (напр. ленти на грешката, Box plot)}
\item Минимизирайте безпорядъка.
\end{itemize}

Вижте файла \texttt{guide\_to\_visualising.pdf} за допълнителна информация и насоки.

\begin{figure}[htb]
    \centering
    \includegraphics[width=1.0\linewidth]{images/boxplot_finger_distance.pdf}    

    \caption{Среден брой пръсти, открити от сензора за докосване на различни височини над повърхността, усреднено за всички жестове. Пунктираните линии показват
    истинския брой присъстващи пръсти. Box plots включват bootstrapped несигурност за медианата. Ясно е, че устройството е склонно към
    подценяване на броя на пръстите, особено при по-високи $z$ разстояния.
    }

    % use the notation fig:name to cross reference a figure
    \label{fig:boxplot} 
\end{figure}

\subsection{TODO-та}

\todo{Напиши цялата документация на дипломната работа}

\subsection{Цитати}

\cleanchapterquote{Абсолютно всички цитати на известни личности в Интернет са 100\% верни.}{Алберт Айнщайн}{(ТУЕС-ар)}

\subsection{Източници и ръководства за стил}
Има много ръководства за добро писане на български и английски език. Не е
задължително да ги прочетете, но те ще подобрят начина, по който пишете.

\begin{itemize}
    \item
    \emph{Как да напишем страхотна научна статия} \cite{Pey17} (\textbf{препоръчително}, въпреки че не пишете научна статия)
    \item
    \emph{Как да пишем със стил} \cite{Von80}. Кратко и лесно за четене. Достъпно онлайн.
    \item
    \emph{Стил: Основите на яснотата и изяществото} \cite{Wil09} Много популярно съвременно ръководство за английски стил.
    \item
    \emph{Политика и английският език} \cite{Orw68} Известно есе за ефективно и ясно писане на английски.
    \item
    \emph{Елементите на стила} \cite{StrWhi07} Остаряло и американско, но класическо.
    \item
    \emph{Усетът за стил} \cite{Pin15} Отлично, макар и доста задълбочено.
\end{itemize}

\subsubsection{Стилове на цитиране}

\begin{itemize}
\item Ако се позовавате на източник като съществително, цитирайте го така: ,,\citet{Orw68} разглежда ролята на езика в политическата мисъл.''
\item Ако се позовавате имплицитно на източници, използвайте: ,,Има много добри книги за писане \citep{Orw68, Wil09, Pin15}.''
\end{itemize}

Пълно ръководство за добри практики при цитиране от Питър Коксхед е достъпно тук: \url{http://www.cs.bham.ac.uk/~pxc/refs/index.html}.
Ако не сте сигурни как да цитирате онлайн източници, вижте \citet{UNSWWebsite}.
\footnote{Понякога е по-подходящо да се посочи онлайн ресурс като \url{https://developer.android.com/studio}
в бележка под линия, отколкото да се включва като формална референция.}

\subsection{Предупреждение за плагиатство}

\begin{highlight_title}{ВНИМАНИЕ}
    
    Ако включвате материал от други източници без пълно и коректно цитиране, извършвате плагиатство. Наказанията за плагиатство са тежки.
    Цитирайте всеки включен текст и го посочвайте правилно. Цитирайте ясно всички изображения, фигури и др. в техните описания.
\end{highlight_title}

\subsection{Цитиране на текст}

Когато цитирате дълъг пасаж, използвайте средата \texttt{quote}:

\begin{quote}
    Ако драскате мислите си как да е, читателите със сигурност ще почувстват, че изобщо не ви е грижа за тях. Те ще ви определят като егоманиак или глупак - или още по-лошо, ще спрат да ви четат. Най-осъдителното разкритие, което можете да направите за себе си, е че не знаете кое е интересно и кое не е.
\end{quote} \citep{Von80}

Когато цитирате в текста, както следната забележка на Саймън Пейтън-Джоунс, използвайте кавички ,,Предаването на интуицията е първостепенно, а не второстепенно'' \citep{Pey17}.