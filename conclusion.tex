\chapter{Заключение}
Summarise the whole project for a lazy reader who didn't read the rest (e.g. a prize-awarding committee). This chapter should be short in most dissertations; maybe one to three pages.
\section{Guidance}
\subsection{Таковата}
\begin{itemize}
    \item
        Summarise briefly and fairly.
    \item
        You should be addressing the general problem you introduced in the
        Introduction.        
    \item
        Include summary of concrete results (``the new compiler ran 2x
        faster'')
    \item
        Indicate what future work could be done, but remember: \textbf{you
        won't get credit for things you haven't done}.
\end{itemize}

\section{Summary}
Summarise what you did; answer the general questions you asked in the introduction. What did you achieve? Briefly describe what was built and summarise the evaluation results.

\section{Reflection}
Discuss what went well and what didn't and how you would do things differently if you did this project again.

\section{Future work}
Discuss what you would do if you could take this further -- where would the interesting directions to go next be? (e.g. you got another year to work on it, or you started a company to work on this, or you pursued a PhD on this topic)
