\documentclass[]{tu_coursework}

\begin{document}

%=====================================================================================================
%% Метаданни
\title{Шаблон за курсова работа}
\author{Димитър Рачков}
\admgroup{69}
\faculty{Факултет Електронна Техника и Технологии}
\fac{ФЕТТ}
\facnumber{101218420}
\supervisor{проф. д.т.н. Стилян Гюров}
% Коментирай долния ред, ако няма рецензент
\reviewer{проф. д.т.н. Юлиан Вучков}
\date{\today}
\city{София}

\makeatletter
\hypersetup{					    % настройки на пакета hyperref
	pdftitle={\@title},      	    % 	- заглавие (PDF meta)
	%pdfsubject={\thesisSubject},   % 	- тема (PDF meta)
	pdfauthor={\@author},           % 	- автор (PDF meta)
	plainpages=false,			    % 	-
	pdfborder={0 0 0},			    % 	-
	breaklinks=true,			    % 	- разрешаване на нови редове в линкове
	bookmarksnumbered=true,		    %
	bookmarksopen=true			    %
}
\makeatother

\maketitle

%=====================================================================================================
%% АБСТРАКТ
\begin{abstract}
    Every abstract follows a similar pattern. Motivate; set aims; describe work; explain results.
    \vskip 0.5em
    ``XYZ is bad. This project investigated ABC to determine if it was better. 
    ABC used XXX and YYY to implement ZZZ. This is particularly interesting as XXX and YYY have
    never been used together. It was found that  
    ABC was 20\% better than XYZ, though it caused rabies in half of subjects.''
\end{abstract}

%=====================================================================================================
%% БЛАГОДАРНОСТИ
%\chapter*{Благодарности}
\begin{center}
    \pagebreak
    \hspace{0pt}
    \vfill
    \large
    Благодарности на\\Роси Мандраджийката за дето домъкна уредбата,\\на Ванката и Томи за дето опекоха скарата\\и на Миле Китич.
    \vfill
    \hspace{0pt}
    \pagebreak
\end{center}

%=====================================================================================================
%% СЪДЪРЖАНИЕ
\tableofcontents
\listoffigures
\listoftables

\pagenumbering{arabic}
%=====================================================================================================
%% ГЛАВИ
% Добра структура е всяка глава да бъде в отделен .tex файл.
\chapter*{Увод}
\addcontentsline{toc}{chapter}{Увод}

В увода трябва да се изложи кратко въведение в областта \emph{(максимум 1-2 страници)}.

В края на увода трябва да бъдат поставени целите и задачите на дипломната работа.

\chapter{Background}
\label{kurche}
What did other people do, and how is it relevant to what you want to do?
\section{Guidance}
\begin{itemize}    
    \item
      Don't give a laundry list of references. \Cref{kurche}
    \item
      Tie everything you say to your problem. \cref{kurche}
    \item
      Present an argument.
    \item Think critically; weigh up the contribution of the background and put it in context.    
    \item
      \textbf{Don't write a tutorial}; provide background and cite
      references for further information.
\end{itemize}

%=====================================================================================================
\chapter{Analysis/Requirements}
What is the problem that you want to solve, and how did you arrive at it?
\section{Guidance}
Make it clear how you derived the constrained form of your problem via a clear and logical process. 

The analysis chapter explains the process by which you arrive at a concrete design. In software 
engineering projects, this will include a statement of the requirement capture process and the
derived requirements.

In research projects, it will involve developing a design drawing on
the work established in the background, and stating how the space of possible projects was
sensibly narrowed down to what you have done.

%=====================================================================================================
\chapter{Design}
How is this problem to be approached, without reference to specific\\
implementation details? 
\section{Guidance}
Design should cover the abstract design in such a way that someone else might be able to do what you did, 
but with a different language or library or tool. This might include overall system architecture diagrams,
user interface designs (wireframes/personas/etc.), protocol specifications, algorithms, data set design choices,
among others. Specific languages, technical choices, libraries and such like should not usually appear in the design. These are implementation details.


%=====================================================================================================
\chapter{Implementation}
What did you do to implement this idea, and what technical achievements did you make?
\section{Guidance}
You can't talk about everything. Cover the high level first, then cover important, relevant or impressive details.

\section{General guidance for technical writing}

These points apply to the whole dissertation, not just this chapter.

\subsection{Figures}
\emph{Always} refer to figures included, like Figure \cref{fig:relu}, in the body of the text. Include full, explanatory captions and make sure the figures look good on the page.
You may include multiple figures in one float, as in Figure \cref{fig:synthetic}, using \texttt{subcaption}, which is enabled in the template.

\Crefrange{fig:syn1}{fig:syn2} са грозни.

% Figures are important. Use them well.
\begin{figure}[htb]
    \centering
    \includegraphics[width=0.5\linewidth]{images/relu.pdf}    

    \caption{In figure captions, explain what the reader is looking at: ``A schematic of the rectifying linear unit, where $a$ is the output amplitude,
    $d$ is a configurable dead-zone, and $Z_j$ is the input signal'', as well as why the reader is looking at this: 
    ``It is notable that there is no activation \emph{at all} below 0, which explains our initial results.'' 
    \textbf{Use vector image formats (.pdf) where possible}. Size figures appropriately, and do not make them over-large or too small to read.
    }

    % use the notation fig:name to cross reference a figure
    \label{fig:relu} 
\end{figure}


\begin{figure}[htb] 
    \centering
    \begin{subfigure}[h]{0.45\textwidth}
        \includegraphics[width=\textwidth]{images/synthetic.png}
        \caption{Synthetic image, black on white.}
        \label{fig:syn1}
    \end{subfigure}
    ~ %add desired spacing between images, e. g. ~, \quad, \qquad, \hfill etc. 
      %(or a blank line to force the subfigure onto a new line)
    \begin{subfigure}[h]{0.45\textwidth}
        \includegraphics[width=\textwidth]{images/synthetic_2.png}
        \caption{Synthetic image, white on black.}
        \label{fig:syn2}
    \end{subfigure}
    ~ %add desired spacing between images, e. g. ~, \quad, \qquad, \hfill etc. 
    %(or a blank line to force the subfigure onto a new line)    
    \caption{Synthetic test images for edge detection algorithms. \subref{fig:syn1} shows various gray levels that require an adaptive algorithm. \subref{fig:syn2}
    shows more challenging edge detection tests that have crossing lines. Fusing these into full segments typically requires algorithms like the Hough transform.
    This is an example of using subfigures, with \texttt{subref}s in the caption.
    }\label{fig:synthetic}
\end{figure}

\clearpage

\subsection{Equations}

Equations should be typeset correctly and precisely. Make sure you get parenthesis sizing correct, and punctuate equations correctly 
(the comma is important and goes \textit{inside} the equation block). Explain any symbols used clearly if not defined earlier. 

For example, we might define:
\begin{equation}
    12\hat{f}(\xi) = \frac{1}{2}\left[ \int_{-\infty}^{\infty} f(x) e^{2\pi i x \xi} \right],
\end{equation}    
where $\hat{f}(\xi)$ is the Fourier transform of the time domain signal $f(x)$.

\subsection{Algorithms}
Algorithms can be set using \texttt{algorithm2e}, as in Algorithm \cref{alg:metropolis}.

% NOTE: line ends are denoted by \; in algorithm2e
\begin{algorithm}
    \DontPrintSemicolon
    \KwData{$f_X(x)$, a probability density function returing the density at $x$.\; $\sigma$ a standard deviation specifying the spread of the proposal distribution.\;
    $x_0$, an initial starting condition.}
    \KwResult{$s=[x_1, x_2, \dots, x_n]$, $n$ samples approximately drawn from a distribution with PDF $f_X(x)$.}
    \Begin{
        $s \longleftarrow []$\;
        $p \longleftarrow f_X(x)$\;
        $i \longleftarrow 0$\;
        \While{$i < n$}
        {
            $x^\prime \longleftarrow \mathcal{N}(x, \sigma^2)$\;
            $p^\prime \longleftarrow f_X(x^\prime)$\;
            $a \longleftarrow \frac{p^\prime}{p}$\;
            $r \longleftarrow U(0,1)$\;
            \If{$r<a$}
            {
                $x \longleftarrow x^\prime$\;
                $p \longleftarrow f_X(x)$\;
                $i \longleftarrow i+1$\;
                append $x$ to $s$\;
            }
        }
    }
    
\caption{The Metropolis-Hastings MCMC algorithm for drawing samples from arbitrary probability distributions, 
specialised for normal proposal distributions $q(x^\prime|x) = \mathcal{N}(x, \sigma^2)$. The symmetry of the normal distribution means the acceptance rule takes the simplified form.}\label{alg:metropolis}
\end{algorithm}

\subsection{Tables}

If you need to include tables, like Table \cref{tab:operators}, use a tool like\\ https://www.tablesgenerator.com/ to generate the table as it is
extremely tedious otherwise. 

\begin{table}
    \caption{The standard table of operators in Python, along with their functional equivalents from the \texttt{operator} package. Note that table
    captions go above the table, not below. Do not add additional rules/lines to tables. }\label{tab:operators}
    %\tt 
    \rowcolors{2}{}{gray!3}
    \begin{tabular}{@{}lll@{}}
    %\toprule
    \textbf{Operation}    & \textbf{Syntax}                & \textbf{Function}                            \\ %\midrule % optional rule for header
    Addition              & \texttt{a + b}                          & \texttt{add(a, b)}                                    \\
    Concatenation         & \texttt{seq1 + seq2}                    & \texttt{concat(seq1, seq2)}                           \\
    Containment Test      & \texttt{obj in seq}                     & \texttt{contains(seq, obj)}                           \\
    Division              & \texttt{a / b}                          & \texttt{div(a, b) }  \\
    Division              & \texttt{a / b}                          & \texttt{truediv(a, b) } \\
    Division              & \texttt{a // b}                         & \texttt{floordiv(a, b)}                               \\
    Bitwise And           & \texttt{a \& b}                         & \texttt{and\_(a, b)}                                  \\
    Bitwise Exclusive Or  & \texttt{a \textasciicircum b}           & \texttt{xor(a, b)}                                    \\
    Bitwise Inversion     & \texttt{$\sim$a}                        & \texttt{invert(a)}                                    \\
    Bitwise Or            & \texttt{a | b}                          & \texttt{or\_(a, b)}                                   \\
    Exponentiation        & \texttt{a ** b}                         & \texttt{pow(a, b)}                                    \\
    Identity              & \texttt{a is b}                         & \texttt{is\_(a, b)}                                   \\
    Identity              & \texttt{a is not b}                     & \texttt{is\_not(a, b)}                                \\
    Indexed Assignment    & \texttt{obj{[}k{]} = v}                 & \texttt{setitem(obj, k, v)}                           \\
    Indexed Deletion      & \texttt{del obj{[}k{]}}                 & \texttt{delitem(obj, k)}                              \\
    Indexing              & \texttt{obj{[}k{]}}                     & \texttt{getitem(obj, k)}                              \\
    Left Shift            & \texttt{a \textless{}\textless b}       & \texttt{lshift(a, b)}                                 \\
    Modulo                & \texttt{a \% b}                         & \texttt{mod(a, b)}                                    \\
    Multiplication        & \texttt{a * b}                          & \texttt{mul(a, b)}                                    \\
    Negation (Arithmetic) & \texttt{- a}                            & \texttt{neg(a)}                                       \\
    Negation (Logical)    & \texttt{not a}                          & \texttt{not\_(a)}                                     \\
    Positive              & \texttt{+ a}                            & \texttt{pos(a)}                                       \\
    Right Shift           & \texttt{a \textgreater{}\textgreater b} & \texttt{rshift(a, b)}                                 \\
    Sequence Repetition   & \texttt{seq * i}                        & \texttt{repeat(seq, i)}                               \\
    Slice Assignment      & \texttt{seq{[}i:j{]} = values}          & \texttt{setitem(seq, slice(i, j), values)}            \\
    Slice Deletion        & \texttt{del seq{[}i:j{]}}               & \texttt{delitem(seq, slice(i, j))}                    \\
    Slicing               & \texttt{seq{[}i:j{]}}                   & \texttt{getitem(seq, slice(i, j))}                    \\
    String Formatting     & \texttt{s \% obj}                       & \texttt{mod(s, obj)}                                  \\
    Subtraction           & \texttt{a - b}                          & \texttt{sub(a, b)}                                    \\
    Truth Test            & \texttt{obj}                            & \texttt{truth(obj)}                                   \\
    Ordering              & \texttt{a \textless b}                  & \texttt{lt(a, b)}                                     \\
    Ordering              & \texttt{a \textless{}= b}               & \texttt{le(a, b)}                                     \\
    % \bottomrule
    \end{tabular}
    \end{table}
\subsection{Code}

Avoid putting large blocks of code in the report (more than a page in one block, for example). Use syntax highlighting if possible, as in Listing \cref{lst:callahan}.

\begin{lstlisting}[language=python, float, caption={The algorithm for packing the $3\times 3$ outer-totalistic binary CA successor rule into a 
    $16\times 16\times 16\times 16$ 4 bit lookup table, running an equivalent, notionally 16-state $2\times 2$ CA.}, label=lst:callahan]
    def create_callahan_table(rule="b3s23"):
        """Generate the lookup table for the cells."""        
        s_table = np.zeros((16, 16, 16, 16), dtype=np.uint8)
        birth, survive = parse_rule(rule)

        # generate all 16 bit strings
        for iv in range(65536):
            bv = [(iv >> z) & 1 for z in range(16)]
            a, b, c, d, e, f, g, h, i, j, k, l, m, n, o, p = bv

            # compute next state of the inner 2x2
            nw = apply_rule(f, a, b, c, e, g, i, j, k)
            ne = apply_rule(g, b, c, d, f, h, j, k, l)
            sw = apply_rule(j, e, f, g, i, k, m, n, o)
            se = apply_rule(k, f, g, h, j, l, n, o, p)

            # compute the index of this 4x4
            nw_code = a | (b << 1) | (e << 2) | (f << 3)
            ne_code = c | (d << 1) | (g << 2) | (h << 3)
            sw_code = i | (j << 1) | (m << 2) | (n << 3)
            se_code = k | (l << 1) | (o << 2) | (p << 3)

            # compute the state for the 2x2
            next_code = nw | (ne << 1) | (sw << 2) | (se << 3)

            # get the 4x4 index, and write into the table
            s_table[nw_code, ne_code, sw_code, se_code] = next_code

        return s_table

\end{lstlisting}

%=====================================================================================================
\chapter{Evaluation} 
How good is your solution? How well did you solve the general problem, and what evidence do you have to support that?

\section{Guidance}
\begin{itemize}
    \item
        Ask specific questions that address the general problem.
    \item
        Answer them with precise evidence (graphs, numbers, statistical
        analysis, qualitative analysis).
    \item
        Be fair and be scientific.
    \item
        The key thing is to show that you know how to evaluate your work, not
        that your work is the most amazing product ever.
\end{itemize}

\section{Evidence}
Make sure you present your evidence well. Use appropriate visualisations, 
reporting techniques and statistical analysis, as appropriate. The point is not
to dump all the data you have but to present an argument well supported by evidence gathered.

If you use numerical evidence, specify reasonable numbers of significant digits; don't state ``18.41141\% of users were successful'' if you only had 20 users. If you average \textit{anything}, present both a measure of central tendency (e.g. mean, median) \textit{and} a measure of spread (e.g. standard deviation, min/max, interquartile range).

You can use \texttt{siunitx} to define units, space numbers neatly, and set the precision for the whole LaTeX document. 

% setup siunitx to have two decimal places
\sisetup{
	round-mode = places,
	round-precision = 2
}

For example, these numbers will appear with two decimal places: \num{3.141592}, \num{2.71828}, and this one will appear with reasonable spacing \num{1000000}.



If you use statistical procedures, make sure you understand the process you are using,
and that you check the required assumptions hold in your case. 

If you visualise, follow the basic rules, as illustrated in Figure \cref{fig:boxplot}:
\begin{itemize}
\item Label everything correctly (axis, title, units).
\item Caption thoroughly.
\item Reference in text.
\item \textbf{Include appropriate display of uncertainty (e.g. error bars, Box plot)}
\item Minimize clutter.
\end{itemize}

See the file \texttt{guide\_to\_visualising.pdf} for further information and guidance.

\begin{figure}[htb]
    \centering
    \includegraphics[width=1.0\linewidth]{images/boxplot_finger_distance.pdf}    

    \caption{Average number of fingers detected by the touch sensor at different heights above the surface, averaged over all gestures. Dashed lines indicate
    the true number of fingers present. The Box plots include bootstrapped uncertainty notches for the median. It is clear that the device is biased toward 
    undercounting fingers, particularly at higher $z$ distances.
    }

    % use the notation fig:name to cross reference a figure
    \label{fig:boxplot} 
\end{figure}

\chapter{Заключение}
Summarise the whole project for a lazy reader who didn't read the rest (e.g. a prize-awarding committee). This chapter should be short in most dissertations; maybe one to three pages.
\section{Guidance}
\subsection{Таковата}
\begin{itemize}
    \item
        Summarise briefly and fairly.
    \item
        You should be addressing the general problem you introduced in the
        Introduction.        
    \item
        Include summary of concrete results (``the new compiler ran 2x
        faster'')
    \item
        Indicate what future work could be done, but remember: \textbf{you
        won't get credit for things you haven't done}.
\end{itemize}

\section{Summary}
Summarise what you did; answer the general questions you asked in the introduction. What did you achieve? Briefly describe what was built and summarise the evaluation results.

\section{Reflection}
Discuss what went well and what didn't and how you would do things differently if you did this project again.

\section{Future work}
Discuss what you would do if you could take this further -- where would the interesting directions to go next be? (e.g. you got another year to work on it, or you started a company to work on this, or you pursued a PhD on this topic)


%=====================================================================================================
%% БИБЛИОГРАФИЯ
% Не мести библиографията след приложенията! Бъгясва се и заглавието й изчезва. Нямам си идея защо.
\nocite{soviet_ics}
\nocite{rechniko}
\printbibliography

%=====================================================================================================
%% ПРИЛОЖЕНИЯ
\addcontentsline{toc}{chapter}{Приложения}
\appendix
%=====================================================================================================

Разпечатка на сорс кода на програмата. Ако е твърде голям, да се приложи на CD ROM 

\begin{highlight_title}{ЗАДЪЛЖИТЕЛНО!}
    Всички работни файлове (фигури, алгоритми, графики, таблици, чертежи), както и текста на дипломната работа трябва да се приложат в отделна папка. Сорс кодът и работоспособен изпълним файл на програмната система  трябва да се приложат в отделна папка на CD ROM, който е надписан с имената и випуска на дипломанта и  поставени в джоб на задната корица на подвързаната ДР.
\end{highlight_title}

\chapter{Примери за използване на шаблона}
\section{Примерен дълъг текст}

В съвременния свят на технологиите и дигиталната трансформация, разработването на софтуерни системи се превръща в изключително важна и комплексна задача. Програмните продукти трябва да отговарят на все по-високи изисквания за производителност, сигурност и надеждност, като същевременно предоставят интуитивен и удобен потребителски интерфейс.

Разработката на големи софтуерни системи изисква задълбочено познаване на множество технологии, програмни езици и методологии за разработка. Съвременните разработчици трябва да са запознати с различни архитектурни подходи, шаблони за дизайн и добри практики при писането на код. Освен това, те трябва да могат да работят ефективно в екип, да използват системи за контрол на версиите и да следват установените процеси за разработка на софтуер.

Един от ключовите аспекти при създаването на качествен софтуер е правилното планиране и проектиране на системата. Това включва внимателен анализ на изискванията, създаване на детайлна архитектура и избор на подходящи технологии. Важно е също така да се предвидят възможните проблеми и предизвикателства, които могат да възникнат по време на разработката.

Тестването на софтуера е друг критичен елемент от процеса на разработка. То трябва да бъде извършвано на различни нива - от unit тестове до интеграционни и системни тестове. Автоматизацията на тестването е особено важна при големи проекти, тъй като позволява бързо откриване на проблеми и осигурява по-висока надеждност на крайния продукт.

Сигурността на софтуерните системи е тема, която придобива все по-голямо значение. Разработчиците трябва да са наясно с потенциалните заплахи и да прилагат подходящи мерки за защита. Това включва защита от SQL инжекции, XSS атаки, правилно управление на потребителските данни и криптиране на чувствителна информация.

Оптимизацията на производителността е друг важен аспект при разработката на софтуер. Това включва ефективно използване на системните ресурси, оптимизация на заявките към базата данни, кеширане на често използвани данни и минимизиране на мрежовия трафик. При уеб приложенията особено важно е да се осигури добра производителност при голям брой едновременни потребители.

Документацията на софтуера е често пренебрегван, но изключително важен елемент. Добрата документация улеснява поддръжката и развитието на системата, помага при въвеждането на нови членове в екипа и служи като референция при възникване на проблеми. Тя трябва да включва както техническа документация за разработчици, така и ръководства за потребители.

Непрекъснатата интеграция и доставка (CI/CD) са станали неразделна част от съвременната софтуерна разработка. Те позволяват автоматизиране на процесите по изграждане, тестване и внедряване на софтуера, което води до по-бързо разработване и по-надеждни релийзи.

Поддръжката на софтуера след неговото внедряване е също толкова важна, колкото и самата разработка. Това включва отстраняване на открити бъгове, добавяне на нови функционалности и оптимизации базирани на обратна връзка от потребителите. Важно е да се поддържа баланс между добавянето на нови функции и запазването на стабилността на системата.

Управлението на техническия дълг е друго предизвикателство, с което се сблъскват разработчиците. Понякога, поради времеви ограничения или други фактори, се налага да се правят компромиси с качеството на кода. Важно е тези компромиси да бъдат документирани и планирани за бъдещо оптимизиране.

В заключение, разработката на софтуер е комплексна дейност, която изисква широк спектър от знания и умения. Успешните проекти са резултат от добро планиране, правилен избор на технологии, ефективна работа в екип и стриктно следване на установените добри практики в софтуерното инженерство.

\section{Общи насоки за техническо писане}

Тези точки се отнасят за цялата дипломна работа, не само за тази глава.

\subsection{Фигури}
\emph{Винаги} се позовавайте на включените фигури, като Фигура \cref{fig:relu}, в основния текст. Включвайте пълни, обяснителни надписи и се уверете, че фигурите изглеждат добре на страницата.
Можете да включите множество фигури в един float, както във Фигура \cref{fig:synthetic}, използвайки \texttt{subcaption}, който е активиран в шаблона.


% Figures are important. Use them well.
\begin{figure}[htb]
    \centering
    \includegraphics[width=0.5\linewidth]{images/relu.pdf}    

    \caption{В надписите на фигурите обяснете какво вижда читателят: ,,Схема на изправящата линейна единица, където $a$ е изходната амплитуда,
    $d$ е конфигурируема мъртва зона, а $Z_j$ е входният сигнал'', както и защо читателят гледа това: 
    ,,Забележително е, че няма никаква активация \emph{изобщо} под 0, което обяснява първоначалните ни резултати.'' 
    \textbf{Използвайте векторни формати на изображения (.pdf) където е възможно}. Оразмерявайте фигурите подходящо и не ги правете прекалено големи или твърде малки за четене.
    }

    % use the notation fig:name to cross reference a figure
    \label{fig:relu} 
\end{figure}


\begin{figure}[htb] 
    \centering
    \begin{subfigure}[h]{0.45\textwidth}
        \includegraphics[width=\textwidth]{images/synthetic.png}
        \caption{Синтетично изображение, черно на бяло.}
        \label{fig:syn1}
    \end{subfigure}
    ~ %add desired spacing between images, e. g. ~, \quad, \qquad, \hfill etc. 
      %(or a blank line to force the subfigure onto a new line)
    \begin{subfigure}[h]{0.45\textwidth}
        \includegraphics[width=\textwidth]{images/synthetic_2.png}
        \caption{Синтетично изображение, бяло на черно.}
        \label{fig:syn2}
    \end{subfigure}
    ~ %add desired spacing between images, e. g. ~, \quad, \qquad, \hfill etc. 
    %(or a blank line to force the subfigure onto a new line)    
    \caption{Синтетични тестови изображения за алгоритми за откриване на ръбове. \subref{fig:syn1} показва различни нива на сиво, които изискват адаптивен алгоритъм. \subref{fig:syn2}
    показва по-предизвикателни тестове за откриване на ръбове, които имат пресичащи се линии. Сливането им в пълни сегменти обикновено изисква алгоритми като трансформацията на Хаф.
    Това е пример за използване на подфигури, с \texttt{subref}s в надписа.
    }\label{fig:synthetic}
\end{figure}

\clearpage

\subsection{Уравнения}

Уравненията трябва да бъдат набрани правилно и прецизно. Уверете се, че размерът на скобите е правилен и пунктуацията на уравненията е правилна 
(запетаята е важна и отива \textit{вътре} в блока на уравнението). Обяснете ясно всички използвани символи, ако не са дефинирани по-рано. 

Например, бихме могли да дефинираме:
\begin{equation}
    \hat{f}(\xi) = \frac{1}{2}\left[ \int_{-\infty}^{\infty} f(x) e^{2\pi i x \xi} \right],
\end{equation}    
където $\hat{f}(\xi)$ е Фурие трансформацията на времевия сигнал $f(x)$.

\subsection{Алгоритми}
Алгоритмите могат да бъдат набрани с помощта на \texttt{algorithm2e}, както в Алгоритъм \cref{alg:metropolis}.

% NOTE: line ends are denoted by \; in algorithm2e
\begin{algorithm}
    \DontPrintSemicolon
    \KwData{$f_X(x)$, a probability density function returing the density at $x$.\; $\sigma$ a standard deviation specifying the spread of the proposal distribution.\;
    $x_0$, an initial starting condition.}
    \KwResult{$s=[x_1, x_2, \dots, x_n]$, $n$ samples approximately drawn from a distribution with PDF $f_X(x)$.}
    \Begin{
        $s \longleftarrow []$\;
        $p \longleftarrow f_X(x)$\;
        $i \longleftarrow 0$\;
        \While{$i < n$}
        {
            $x^\prime \longleftarrow \mathcal{N}(x, \sigma^2)$\;
            $p^\prime \longleftarrow f_X(x^\prime)$\;
            $a \longleftarrow \frac{p^\prime}{p}$\;
            $r \longleftarrow U(0,1)$\;
            \If{$r<a$}
            {
                $x \longleftarrow x^\prime$\;
                $p \longleftarrow f_X(x)$\;
                $i \longleftarrow i+1$\;
                append $x$ to $s$\;
            }
        }
    }
    
\caption{Алгоритъмът Metropolis-Hastings MCMC за извличане на проби от произволни вероятностни разпределения, 
специализиран за нормални предложени разпределения $q(x^\prime|x) = \mathcal{N}(x, \sigma^2)$. Симетрията на нормалното разпределение означава, че правилото за приемане приема опростената форма.}\label{alg:metropolis}
\end{algorithm}

\subsection{Таблици}

Ако трябва да включите таблици, като Таблица \cref{tab:operators}, използвайте инструмент като\\ https://www.tablesgenerator.com/ за генериране на таблицата, тъй като е
изключително досадно в противен случай. 

\begin{table}
    \caption{Стандартната таблица с оператори в Python, заедно с техните функционални еквиваленти от пакета \texttt{operator}. Забележете, че надписите
    на таблиците отиват над таблицата, а не под нея. Не добавяйте допълнителни линии/черти към таблиците.}\label{tab:operators}
    %\tt 
    \rowcolors{2}{}{gray!3}
    \begin{tabular}{@{}lll@{}}
    %\toprule
    \textbf{Операция}    & \textbf{Синтаксис}                & \textbf{Функция}                            \\ %\midrule % optional rule for header
    Addition              & \texttt{a + b}                          & \texttt{add(a, b)}                                    \\
    Concatenation         & \texttt{seq1 + seq2}                    & \texttt{concat(seq1, seq2)}                           \\
    Containment Test      & \texttt{obj in seq}                     & \texttt{contains(seq, obj)}                           \\
    Division              & \texttt{a / b}                          & \texttt{div(a, b) }  \\
    Division              & \texttt{a / b}                          & \texttt{truediv(a, b) } \\
    Division              & \texttt{a // b}                         & \texttt{floordiv(a, b)}                               \\
    Bitwise And           & \texttt{a \& b}                         & \texttt{and\_(a, b)}                                  \\
    Bitwise Exclusive Or  & \texttt{a \textasciicircum b}           & \texttt{xor(a, b)}                                    \\
    Bitwise Inversion     & \texttt{$\sim$a}                        & \texttt{invert(a)}                                    \\
    Bitwise Or            & \texttt{a | b}                          & \texttt{or\_(a, b)}                                   \\
    Exponentiation        & \texttt{a ** b}                         & \texttt{pow(a, b)}                                    \\
    Identity              & \texttt{a is b}                         & \texttt{is\_(a, b)}                                   \\
    Identity              & \texttt{a is not b}                     & \texttt{is\_not(a, b)}                                \\
    Indexed Assignment    & \texttt{obj{[}k{]} = v}                 & \texttt{setitem(obj, k, v)}                           \\
    Indexed Deletion      & \texttt{del obj{[}k{]}}                 & \texttt{delitem(obj, k)}                              \\
    Indexing              & \texttt{obj{[}k{]}}                     & \texttt{getitem(obj, k)}                              \\
    Left Shift            & \texttt{a \textless{}\textless b}       & \texttt{lshift(a, b)}                                 \\
    Modulo                & \texttt{a \% b}                         & \texttt{mod(a, b)}                                    \\
    Multiplication        & \texttt{a * b}                          & \texttt{mul(a, b)}                                    \\
    Negation (Arithmetic) & \texttt{- a}                            & \texttt{neg(a)}                                       \\
    Negation (Logical)    & \texttt{not a}                          & \texttt{not\_(a)}                                     \\
    Positive              & \texttt{+ a}                            & \texttt{pos(a)}                                       \\
    Right Shift           & \texttt{a \textgreater{}\textgreater b} & \texttt{rshift(a, b)}                                 \\
    Sequence Repetition   & \texttt{seq * i}                        & \texttt{repeat(seq, i)}                               \\
    Slice Assignment      & \texttt{seq{[}i:j{]} = values}          & \texttt{setitem(seq, slice(i, j), values)}            \\
    Slice Deletion        & \texttt{del seq{[}i:j{]}}               & \texttt{delitem(seq, slice(i, j))}                    \\
    Slicing               & \texttt{seq{[}i:j{]}}                   & \texttt{getitem(seq, slice(i, j))}                    \\
    String Formatting     & \texttt{s \% obj}                       & \texttt{mod(s, obj)}                                  \\
    Subtraction           & \texttt{a - b}                          & \texttt{sub(a, b)}                                    \\
    Truth Test            & \texttt{obj}                            & \texttt{truth(obj)}                                   \\
    Ordering              & \texttt{a \textless b}                  & \texttt{lt(a, b)}                                     \\
    Ordering              & \texttt{a \textless{}= b}               & \texttt{le(a, b)}                                     \\
    % \bottomrule
    \end{tabular}
    \end{table}
\subsection{Код}

Избягвайте да поставяте големи блокове код в отчета (повече от страница в един блок, например). Използвайте оцветяване на синтаксиса, ако е възможно, както в Листинг \cref{lst:callahan}.

\begin{lstlisting}[language=python, float, caption={The algorithm for packing the $3\times 3$ outer-totalistic binary CA successor rule into a 
    $16\times 16\times 16\times 16$ 4 bit lookup table, running an equivalent, notionally 16-state $2\times 2$ CA.}, label=lst:callahan]
    def create_callahan_table(rule="b3s23"):
        """Generate the lookup table for the cells."""        
        s_table = np.zeros((16, 16, 16, 16), dtype=np.uint8)
        birth, survive = parse_rule(rule)

        # generate all 16 bit strings
        for iv in range(65536):
            bv = [(iv >> z) & 1 for z in range(16)]
            a, b, c, d, e, f, g, h, i, j, k, l, m, n, o, p = bv

            # compute next state of the inner 2x2
            nw = apply_rule(f, a, b, c, e, g, i, j, k)
            ne = apply_rule(g, b, c, d, f, h, j, k, l)
            sw = apply_rule(j, e, f, g, i, k, m, n, o)
            se = apply_rule(k, f, g, h, j, l, n, o, p)

            # compute the index of this 4x4
            nw_code = a | (b << 1) | (e << 2) | (f << 3)
            ne_code = c | (d << 1) | (g << 2) | (h << 3)
            sw_code = i | (j << 1) | (m << 2) | (n << 3)
            se_code = k | (l << 1) | (o << 2) | (p << 3)

            # compute the state for the 2x2
            next_code = nw | (ne << 1) | (sw << 2) | (se << 3)

            # get the 4x4 index, and write into the table
            s_table[nw_code, ne_code, sw_code, se_code] = next_code

        return s_table

\end{lstlisting}

%=====================================================================================================

\subsection{Доказателства}
Уверете се, че представяте доказателствата си добре. Използвайте подходящи визуализации, 
техники за докладване и статистически анализ, според случая. Целта не е
да изсипете всички данни, които имате, а да представите добре подкрепен с доказателства аргумент.

Ако използвате числови доказателства, посочете разумен брой значещи цифри; не казвайте ,,18.41141\% от потребителите са били успешни'', ако сте имали само 20 потребители. Ако усреднявате \textit{каквото и да е}, представете както мярка за централна тенденция (напр. средна стойност, медиана), \textit{така и} мярка за разпръскване (напр. стандартно отклонение, мин/макс, интерквартилен диапазон).

Можете да използвате \texttt{siunitx} за дефиниране на единици, подреждане на числата спретнато и задаване на прецизност за целия LaTeX документ. 

% setup siunitx to have two decimal places
\sisetup{
	round-mode = places,
	round-precision = 2
}

Например, тези числа ще се появят с два десетични знака: \num{3.141592}, \num{2.71828}, а това ще се появи с разумно разстояние \num{1000000}.

Ако използвате статистически процедури, уверете се, че разбирате процеса, който използвате,
и че проверявате дали необходимите предположения са валидни във вашия случай. 

Ако визуализирате, следвайте основните правила, както е илюстрирано във Фигура \cref{fig:boxplot}:
\begin{itemize}
\item Надпишете всичко правилно (оси, заглавие, единици).
\item Надпишете подробно.
\item Цитирайте в текста.
\item \textbf{Включете подходящо показване на несигурността (напр. ленти на грешката, Box plot)}
\item Минимизирайте безпорядъка.
\end{itemize}

Вижте файла \texttt{guide\_to\_visualising.pdf} за допълнителна информация и насоки.

\begin{figure}[htb]
    \centering
    \includegraphics[width=1.0\linewidth]{images/boxplot_finger_distance.pdf}    

    \caption{Среден брой пръсти, открити от сензора за докосване на различни височини над повърхността, усреднено за всички жестове. Пунктираните линии показват
    истинския брой присъстващи пръсти. Box plots включват bootstrapped несигурност за медианата. Ясно е, че устройството е склонно към
    подценяване на броя на пръстите, особено при по-високи $z$ разстояния.
    }

    % use the notation fig:name to cross reference a figure
    \label{fig:boxplot} 
\end{figure}

\subsection{TODO-та}

\todo{Напиши цялата документация на дипломната работа}

\subsection{Цитати}

\cleanchapterquote{Абсолютно всички цитати на известни личности в Интернет са 100\% верни.}{Алберт Айнщайн}{(ТУЕС-ар)}

\subsection{Източници и ръководства за стил}
Има много ръководства за добро писане на български и английски език. Не е
задължително да ги прочетете, но те ще подобрят начина, по който пишете.

\begin{itemize}
    \item
    \emph{Как да напишем страхотна научна статия} \cite{Pey17} (\textbf{препоръчително}, въпреки че не пишете научна статия)
    \item
    \emph{Как да пишем със стил} \cite{Von80}. Кратко и лесно за четене. Достъпно онлайн.
    \item
    \emph{Стил: Основите на яснотата и изяществото} \cite{Wil09} Много популярно съвременно ръководство за английски стил.
    \item
    \emph{Политика и английският език} \cite{Orw68} Известно есе за ефективно и ясно писане на английски.
    \item
    \emph{Елементите на стила} \cite{StrWhi07} Остаряло и американско, но класическо.
    \item
    \emph{Усетът за стил} \cite{Pin15} Отлично, макар и доста задълбочено.
\end{itemize}

\subsubsection{Стилове на цитиране}

\begin{itemize}
\item Ако се позовавате на източник като съществително, цитирайте го така: ,,\citet{Orw68} разглежда ролята на езика в политическата мисъл.''
\item Ако се позовавате имплицитно на източници, използвайте: ,,Има много добри книги за писане \citep{Orw68, Wil09, Pin15}.''
\end{itemize}

Пълно ръководство за добри практики при цитиране от Питър Коксхед е достъпно тук: \url{http://www.cs.bham.ac.uk/~pxc/refs/index.html}.
Ако не сте сигурни как да цитирате онлайн източници, вижте \citet{UNSWWebsite}.
\footnote{Понякога е по-подходящо да се посочи онлайн ресурс като \url{https://developer.android.com/studio}
в бележка под линия, отколкото да се включва като формална референция.}

\subsection{Предупреждение за плагиатство}

\begin{highlight_title}{ВНИМАНИЕ}
    
    Ако включвате материал от други източници без пълно и коректно цитиране, извършвате плагиатство. Наказанията за плагиатство са тежки.
    Цитирайте всеки включен текст и го посочвайте правилно. Цитирайте ясно всички изображения, фигури и др. в техните описания.
\end{highlight_title}

\subsection{Цитиране на текст}

Когато цитирате дълъг пасаж, използвайте средата \texttt{quote}:

\begin{quote}
    Ако драскате мислите си как да е, читателите със сигурност ще почувстват, че изобщо не ви е грижа за тях. Те ще ви определят като егоманиак или глупак - или още по-лошо, ще спрат да ви четат. Най-осъдителното разкритие, което можете да направите за себе си, е че не знаете кое е интересно и кое не е.
\end{quote} \citep{Von80}

Когато цитирате в текста, както следната забележка на Саймън Пейтън-Джоунс, използвайте кавички ,,Предаването на интуицията е първостепенно, а не второстепенно'' \citep{Pey17}.

\end{document}
